\documentclass[aspectratio=169]{beamer}
\usetheme{Madrid}
\usecolortheme{default}

% Remove navigation symbols and footer
\setbeamertemplate{navigation symbols}{}
\setbeamertemplate{footline}{}

% Packages
\usepackage{graphicx}
\usepackage{booktabs}
\usepackage{amsmath}
\usepackage{tikz}
\usepackage{xcolor}
\usepackage{hyperref}

% Custom colors
\definecolor{darkblue}{RGB}{0,51,102}
\definecolor{lightblue}{RGB}{102,178,255}

% Graphics path
\graphicspath{{../results/}{../results/2.3-outlier-detection/}{../results/2.5-correlation-analysis/}{../results/3normalization/}{../results/4pca/}{./results/clustering_images/}}

% Title information
\title[Music Genre Discovery]{Unsupervised Music Genre Discovery Using Audio Feature Learning}
\author{Anirudh Sharma \\ Roll No.: 22dcs002}
\institute[NIT Hamirpur]{
    Department of Computer Science and Engineering \\
    National Institute of Technology Hamirpur
}
\date{}

\begin{document}

%==============================================================================
% TITLE SLIDE
%==============================================================================
\begin{frame}[plain]
    \begin{center}
        \vspace*{0.8cm}
        {\LARGE \textbf{Unsupervised Music Genre Discovery}}\\[0.2cm]
        {\Large \textbf{Using Audio Feature Learning}}\\[0.4cm]

        \IfFileExists{../nith_logo.png}{
            \includegraphics[width=0.17\textwidth]{../nith_logo.png}
        }{
            \fbox{\parbox{0.17\textwidth}{\centering NITH\\Logo}}
        }\\[0.4cm]

        {\small Department of Computer Science and Engineering}\\[0.1cm]
        {\small \textbf{National Institute of Technology Hamirpur}}\\[0.2cm]
        {\footnotesize Machine Learning (CS-652)}\\
        {\footnotesize Semester VII}\\[0.5cm]
    \end{center}
    
    \vfill
    \begin{minipage}[b]{\textwidth}
        \tiny
        \begin{tabular}{@{}p{0.48\textwidth}@{}p{0.04\textwidth}@{}p{0.48\textwidth}@{}}
            Presented by: Anirudh Sharma (22dcs002) & & \hfill Presented to: Dr. Kamlesh Datta
        \end{tabular}
    \end{minipage}
    \vspace{0.2cm}
\end{frame}

%==============================================================================
% TABLE OF CONTENTS
%==============================================================================
\begin{frame}{Presentation Outline}
    \tableofcontents
\end{frame}

%==============================================================================
% SECTION 1: INTRODUCTION
%==============================================================================
\section{Introduction}

\begin{frame}{Introduction}
    \frametitle{Introduction}
    
    \begin{columns}[T]
        \column{0.48\textwidth}
        \small
        The exponential growth of digital music platforms has created massive repositories of unlabeled audio data, making manual organization infeasible at scale. This project presents a comprehensive comparative analysis of \textbf{four unsupervised learning algorithms}—K-Means, K-Medoids, Gaussian Mixture Models (GMM), and Spectral Clustering.
        
        \vspace{0.3cm}
        
        We evaluate these algorithms across datasets ranging from 500 to 25,000 samples, incorporating robust preprocessing (outlier detection, StandardScaler, PCA) and comprehensive evaluation using six performance metrics. High-dimensional clusters are visualized using t-SNE to analyze genre separation and cohesion.
        
        \column{0.48\textwidth}
        \begin{figure}
            \centering
            \IfFileExists{../images/spectrogram.png}{
                \includegraphics[width=\textwidth]{../images/spectrogram.png}
            }{
                \fbox{\parbox{0.9\textwidth}{\centering Audio waveform and\\spectrogram representation}}
            }
            \caption{Audio waveform and spectrogram representation}
        \end{figure}
    \end{columns}
\end{frame}

\begin{frame}{What are Music Genres?}
    \begin{columns}
        \column{0.5\textwidth}
        \textbf{Definition}
        \begin{itemize}
            \item Musical categories based on shared characteristics
            \item Defined by instrumentation, rhythm, harmony, and cultural context
            \item Evolve over time and across cultures
        \end{itemize}
        
        \vspace{0.3cm}
        
        \textbf{GTZAN Genre Labels (Our Baseline)}
        \begin{itemize}
            \item We use \textbf{10 genre clusters} from GTZAN:
            \begin{itemize}
                \footnotesize
                \item Blues, Classical, Country
                \item Disco, Hip-hop, Jazz
                \item Metal, Pop, Reggae, Rock
            \end{itemize}
            \item Rock (subgenre): Hard Rock, Punk Rock, Progressive Rock
            \item Hip-Hop: Trap, Boom Bap, Lo-Fi
        \end{itemize}
        
        \column{0.5\textwidth}
        \textbf{Genre Subtypes \& Complexity}
        \begin{itemize}
            \item Over 1,000+ documented subgenres
            \item Hierarchical relationships (parent-child)
            \item Genre fusion and cross-pollination
        \end{itemize}
        
        \vspace{0.3cm}
        
        \begin{alertblock}{Key Challenges}
            \begin{itemize}
                \item \textbf{Subjectivity}: Genre labels vary across listeners
                \item \textbf{Overlap}: Songs span multiple genres
                \item \textbf{Evolution}: Genres constantly change
                \item \textbf{Ambiguity}: Fuzzy boundaries between styles
                \item \textbf{Scale}: Millions of unlabeled tracks
            \end{itemize}
        \end{alertblock}
    \end{columns}
\end{frame}

%==============================================================================
% SECTION 2: LITERATURE REVIEW
%==============================================================================
\section{Literature Review}

\begin{frame}{Recent Advances in Music Genre Analysis}
    \begin{table}[h]
        \centering
        \scriptsize
        \begin{tabular}{p{2.8cm}p{1.5cm}p{3.5cm}p{2.8cm}}
            \toprule
            \textbf{Study} & \textbf{Year} & \textbf{Method} & \textbf{Key Finding} \\
            \midrule
            Singh et al. & 2024 & Unsupervised Raga discovery & Novel class identification in Indian music \\
            Kumar et al. & 2024 & K-Means clustering & Effective for recommendation systems \\
            Ma et al. & 2023 & Speech SSL for music & Cross-domain transfer learning \\
            Wang et al. & 2023 & Angular contrastive loss & Improved embedding separation \\
            Chong et al. & 2023 & Masked spectrogram & Captures temporal dependencies \\
            Tzanetakis \& Cook & 2002 & GTZAN benchmark & Foundational feature engineering \\
            \bottomrule
        \end{tabular}
    \end{table}
    
    \vspace{0.3cm}
    
    \begin{columns}
        \column{0.48\textwidth}
        \textbf{Research Gap}
        \begin{itemize}
            \footnotesize
            \item Limited cross-cultural validation
            \item Few systematic algorithm comparisons
            \item Lack of unified evaluation frameworks
        \end{itemize}
        
        \column{0.48\textwidth}
        \textbf{Our Contribution}
        \begin{itemize}
            \footnotesize
            \item 5 diverse datasets (Western + Indian)
            \item 4 algorithms with 6 metrics
            \item Reproducible experimental pipeline
        \end{itemize}
    \end{columns}
\end{frame}

%==============================================================================
% SECTION 3: METHODOLOGY
%==============================================================================
\section{Methodology}

\begin{frame}{Methodology}
    \begin{center}
        \IfFileExists{../images/flow-diagram.png}{
            \includegraphics[width=0.9\textwidth]{../images/flow-diagram.png}
        }{
            \fbox{\parbox{0.8\textwidth}{\centering Flow Diagram\\Not Found}}
        }
    \end{center}
\end{frame}

%==============================================================================
% SECTION 4: DATASET SELECTION
%==============================================================================
\section{Dataset Selection}

\begin{frame}{Dataset Collection and Analysis}
    \begin{table}[h]
        \centering
        \small
        \begin{tabular}{lcccc}
            \toprule
            \textbf{Dataset} & \textbf{Tracks} & \textbf{Genres} & \textbf{Duration} & \textbf{Balance} \\
            \midrule
            Indian Regional & 500 & 5 & 45s & Perfect \\
            GTZAN & 1,000 & 10 & 30s & Perfect \\
            FMA Small & 8,000 & 8 & 30s & Balanced \\
            Ludwig & 11,300 & 10 & 30s & Mixed \\
            FMA Medium & 17,000 & 16 & 30s & Unbalanced \\
            \midrule
            \textbf{Total} & \textbf{37,800} & \textbf{49} & -- & -- \\
            \bottomrule
        \end{tabular}
    \end{table}
    
    \vspace{0.3cm}
    
    \begin{columns}
        \column{0.48\textwidth}
        \textbf{Genre Coverage}
        \begin{itemize}
            \footnotesize
            \item \textbf{Western}: blues, classical, country, disco, hip-hop, jazz, metal, pop, reggae, rock
            \item \textbf{Indian}: Bollypop, Carnatic, Ghazal, Semiclassical, Sufi
        \end{itemize}
        
        \column{0.48\textwidth}
        \textbf{Dataset Diversity}
        \begin{itemize}
            \footnotesize
            \item Size: 500 to 17,000 tracks
            \item Cultural: Western + Indian traditions
            \item Balance: Perfect to unbalanced
            \item Source: Kaggle, FMA Archive
        \end{itemize}
    \end{columns}
\end{frame}

%==============================================================================
% SECTION 5: FEATURE EXTRACTION
%==============================================================================
\section{Feature Extraction}

\begin{frame}{Feature Extraction}
    \begin{columns}[T]
        \column{0.48\textwidth}
        \textbf{69 Features Extracted (Librosa 0.11.0)}
        
        \vspace{0.2cm}
        
        \textbf{1. Spectral Features (4)}
        \begin{itemize}
            \footnotesize
            \item Spectral Centroid (brightness)
            \item Spectral Rolloff (85\% energy)
            \item Zero-Crossing Rate (noisiness)
            \item RMS Energy (amplitude)
        \end{itemize}
        
        \vspace{0.2cm}
        
        \textbf{2. MFCCs (40)}
        \begin{itemize}
            \footnotesize
            \item 20 coefficients × (mean + std)
            \item Timbral characterization
            \item Human auditory perception model
        \end{itemize}
        
        \vspace{0.2cm}
        
        \textbf{3. Chromagrams (24)}
        \begin{itemize}
            \footnotesize
            \item 12 pitch classes × (mean + std)
            \item Harmonic content (C-B)
        \end{itemize}
        
        \column{0.48\textwidth}
        \textbf{4. Tempo (1)}
        \begin{itemize}
            \footnotesize
            \item BPM via beat tracking
        \end{itemize}
        
        \vspace{0.3cm}
        
        \begin{block}{Extraction Settings}
            \footnotesize
            \begin{itemize}
                \item \textbf{Sample Rate}: 22,050 Hz
                \item \textbf{Window}: 2048 samples (~93ms)
                \item \textbf{Hop Length}: 512 samples (~23ms)
                \item \textbf{Processing}: CPU-based
            \end{itemize}
        \end{block}
        
        \vspace{0.3cm}
        
        \begin{alertblock}{Success Rate}
            \footnotesize
            \textbf{99.94\%} extraction success\\
            37,778 / 37,800 tracks processed\\
            Only 22 files failed (corruption)
        \end{alertblock}
    \end{columns}
\end{frame}

\begin{frame}{Feature Extraction Results}
    \begin{table}[h]
        \centering
        \begin{tabular}{lcccc}
            \toprule
            \textbf{Dataset} & \textbf{Total} & \textbf{Success} & \textbf{Failed} & \textbf{Rate} \\
            \midrule
            Indian Regional & 500 & 500 & 0 & 100.00\% \\
            GTZAN & 1,000 & 999 & 1 & 99.90\% \\
            FMA Small & 8,000 & 7,997 & 3 & 99.96\% \\
            Ludwig & 11,300 & 11,294 & 6 & 99.95\% \\
            FMA Medium & 17,000 & 16,988 & 12 & 99.93\% \\
            \midrule
            \textbf{Combined} & \textbf{37,800} & \textbf{37,778} & \textbf{22} & \textbf{99.94\%} \\
            \bottomrule
        \end{tabular}
    \end{table}
    
    \vspace{0.3cm}
    
    \begin{columns}
        \column{0.48\textwidth}
        \textbf{Failure Analysis}
        \begin{itemize}
            \footnotesize
            \item File corruption: 10 files
            \item Unsupported codec: 7 files
            \item Incomplete downloads: 5 files
        \end{itemize}
        
        \column{0.48\textwidth}
        \textbf{Quality Assurance}
        \begin{itemize}
            \footnotesize
            \item Zero NaN values detected
            \item Zero infinite values
            \item All 69 features validated
        \end{itemize}
    \end{columns}
\end{frame}

%==============================================================================
% SECTION 6: DESCRIPTIVE ANALYSIS
%==============================================================================
\section{Descriptive Analysis}

\begin{frame}{Descriptive Analysis Overview}
    \begin{columns}
        \column{0.48\textwidth}
        \textbf{Purpose}
        \begin{itemize}
            \footnotesize
            \item Understand data quality and characteristics
            \item Identify patterns and anomalies
            \item Justify preprocessing decisions
            \item Guide normalization and dimensionality reduction
        \end{itemize}
        
        \vspace{0.3cm}
        
        \textbf{Four-Phase Analysis}
        \begin{enumerate}
            \footnotesize
            \item \textbf{Data Integrity}: Validate completeness and cleanliness
            \item \textbf{Outlier Detection}: Identify extreme values using IQR
            \item \textbf{Distribution Analysis}: Assess skewness and normality
            \item \textbf{Correlation Analysis}: Examine feature relationships
        \end{enumerate}
        
        \column{0.48\textwidth}
        \textbf{Key Questions Addressed}
        \begin{itemize}
            \footnotesize
            \item Are there missing or corrupted values?
            \item Do outliers represent errors or genuine diversity?
            \item Are features normally distributed?
            \item Which features are highly correlated?
            \item Is dimensionality reduction needed?
        \end{itemize}
        
        \vspace{0.3cm}
        
        \begin{block}{Outcome}
            \footnotesize
            Analysis reveals need for:
            \begin{itemize}
                \scriptsize
                \item StandardScaler normalization (scale differences)
                \item PCA reduction (correlated features, 69D→42D)
                \item No transformation (acceptable skewness)
            \end{itemize}
        \end{block}
    \end{columns}
\end{frame}

\begin{frame}{Phase 1: Data Integrity Validation}
    \begin{columns}[T]
        \column{0.55\textwidth}
        \textbf{Quality Checks Performed}
        
        \begin{table}[h]
            \centering
            \footnotesize
            \begin{tabular}{lccc}
                \toprule
                \textbf{Check} & \textbf{Count} & \textbf{Action} \\
                \midrule
                NaN values & 0 & ✓ None \\
                Infinite values & 0 & ✓ None \\
                Silent files & 4 & ✗ Removed \\
                \midrule
                \textbf{Valid tracks} & \textbf{37,774} & \textbf{99.99\%} \\
                \bottomrule
            \end{tabular}
        \end{table}
        
        \vspace{0.3cm}
        
        \textbf{Silent/Corrupt File Detection}
        \begin{itemize}
            \footnotesize
            \item Threshold: spectral features $<$ 0.001
            \item FMA Small: 1 file removed
            \item FMA Medium: 2 files removed
            \item Ludwig: 1 file removed
        \end{itemize}
        
        \column{0.42\textwidth}
        \begin{block}{Validation Metrics}
            \footnotesize
            \begin{itemize}
                \item \textbf{Original tracks}: 37,778
                \item \textbf{Removed}: 4 (0.011\%)
                \item \textbf{Final dataset}: 37,774
                \item \textbf{Cleanliness}: 99.99\%
            \end{itemize}
        \end{block}
        
        \vspace{0.3cm}
        
        \begin{alertblock}{Key Finding}
            \footnotesize
            Exceptional data quality with robust Librosa extraction. Only corruption-related failures, no feature computation errors.
        \end{alertblock}
    \end{columns}
\end{frame}

\begin{frame}{Phase 2: Outlier Detection (IQR Method)}
    \begin{columns}[T]
        \column{0.55\textwidth}
        \begin{figure}
            \centering
            \includegraphics[width=\textwidth]{2.3-outlier-detection/box-gtzan.png}
            \caption{GTZAN boxplots: tempo, RMS, spectral centroid, ZCR}
        \end{figure}
        
        \column{0.42\textwidth}
        \begin{table}[h]
            \centering
            \scriptsize
            \begin{tabular}{lcc}
                \toprule
                \textbf{Feature} & \textbf{Count} & \textbf{Rate} \\
                \midrule
                Tempo & 358 & 0.95\% \\
                RMS & 326 & 0.86\% \\
                Centroid & 220 & 0.58\% \\
                ZCR & 638 & 1.69\% \\
                \bottomrule
            \end{tabular}
        \end{table}
        
        \vspace{0.2cm}
        
        \textbf{Decision: Retain All}
        \begin{itemize}
            \footnotesize
            \item All rates $<$ 2\% (LOW severity)
            \item Represent genuine musical diversity
            \item No evidence of measurement errors
        \end{itemize}
    \end{columns}
\end{frame}

\begin{frame}{Outlier Comparison Across Datasets}
    \begin{figure}
        \centering
        \includegraphics[width=0.6\textwidth]{2.3-outlier-detection/outlier_comparison.png}
        \caption{Outlier percentages: ZCR shows highest variability (1.69\%), spectral centroid lowest (0.58\%)}
    \end{figure}
\end{frame}

\begin{frame}{Phase 3: Distribution \& Skewness Analysis}
    \begin{columns}
        \column{0.48\textwidth}
        \textbf{Skewness Classification}
        
        \begin{table}[h]
            \centering
            \footnotesize
            \begin{tabular}{lcc}
                \toprule
                \textbf{Severity} & \textbf{Count} & \textbf{\%} \\
                \midrule
                HIGH ($\geq$ 1.0) & 11 & 16.9\% \\
                MODERATE (0.5-1.0) & 35 & 53.8\% \\
                LOW (< 0.5) & 19 & 29.2\% \\
                \midrule
                \textbf{Total} & \textbf{65} & \textbf{100\%} \\
                \bottomrule
            \end{tabular}
        \end{table}
        
        \vspace{0.2cm}
        
        \textbf{Key Features (Acceptable)}
        \begin{itemize}
            \footnotesize
            \item Spectral rolloff: -0.043
            \item Spectral centroid: 0.286
            \item Tempo: 0.429
        \end{itemize}
        
        \column{0.48\textwidth}
        \textbf{Decision: No Transformation}
        \begin{itemize}
            \footnotesize
            \item 70.7\% features moderate-high skew
            \item Core spectral features near-Gaussian
            \item Preserve interpretability
            \item Avoid transformation artifacts
        \end{itemize}
        
        \vspace{0.3cm}
        
        \begin{alertblock}{Rationale}
            \footnotesize
            Logarithmic transformation would affect interpretability and introduce artifacts. StandardScaler normalization handles scale differences effectively for clustering.
        \end{alertblock}
    \end{columns}
\end{frame}

\begin{frame}{Phase 4: Correlation Analysis}
    \begin{columns}[T]
        \column{0.55\textwidth}
        \begin{figure}
            \centering
            \includegraphics[width=\textwidth]{2.5-correlation-analysis/correlation_mfcc_mean_combined.png}
            \caption{MFCC correlation matrix (37,774 tracks)}
        \end{figure}
        
        \column{0.42\textwidth}
        \begin{table}[h]
            \centering
            \scriptsize
            \begin{tabular}{lcc}
                \toprule
                \textbf{Dataset} & \textbf{Mean r} & \textbf{Pairs$>$0.7} \\
                \midrule
                GTZAN & 0.077 & 13 \\
                FMA Small & 0.247 & 0 \\
                FMA Medium & 0.246 & 0 \\
                Indian & 0.155 & 0 \\
                Ludwig & 0.212 & 0 \\
                \bottomrule
            \end{tabular}
        \end{table}
        
        \vspace{0.2cm}
        
        \textbf{Key Findings}
        \begin{itemize}
            \footnotesize
            \item Adjacent MFCCs show correlation
            \item Consistent patterns across datasets
            \item Validates unified PCA approach
        \end{itemize}
    \end{columns}
\end{frame}

%==============================================================================
% SECTION 7: DATA PREPROCESSING
%==============================================================================
\section{Data Preprocessing}

\begin{frame}{Normalization: StandardScaler}
    \begin{columns}[T]
        \column{0.48\textwidth}
        \textbf{Z-Score Normalization}
        
        \begin{equation*}
            z = \frac{x - \mu}{\sigma}
        \end{equation*}
        
        \vspace{0.2cm}
        
        \textbf{Why Normalize?}
        \begin{itemize}
            \footnotesize
            \item Features on different scales:
            \begin{itemize}
                \scriptsize
                \item Tempo: 0-200 BPM
                \item Spectral centroid: 0-8000 Hz
                \item Chroma: 0-1
            \end{itemize}
            \item Distance-based algorithms sensitive
            \item Equal feature contribution
        \end{itemize}
        
        \vspace{0.2cm}
        
        \textbf{Verification Results}
        \begin{itemize}
            \footnotesize
            \item All features: μ = 0.0000
            \item All features: σ = 1.0000
            \item Zero NaN/Inf post-normalization
        \end{itemize}
        
        \column{0.48\textwidth}
        \begin{table}[h]
            \centering
            \scriptsize
            \begin{tabular}{lcc}
                \toprule
                \textbf{Dataset} & \textbf{Tracks} & \textbf{Features} \\
                \midrule
                GTZAN & 999 & 69 \\
                FMA Small & 7,996 & 70 \\
                FMA Medium & 16,986 & 70 \\
                Ludwig & 11,293 & 69 \\
                Indian & 500 & 69 \\
                \midrule
                \textbf{Total} & \textbf{37,774} & \textbf{69-70} \\
                \bottomrule
            \end{tabular}
        \end{table}
        
        \vspace{0.3cm}
        
        \begin{block}{Impact}
            \footnotesize
            \begin{itemize}
                \item ✓ Scale-invariant features
                \item ✓ Improved K-Means convergence
                \item ✓ Better distance metrics
                \item ✓ PCA prerequisite satisfied
            \end{itemize}
        \end{block}
    \end{columns}
\end{frame}

\begin{frame}{Normalization Visual Comparison}
    \begin{figure}
        \centering
        \includegraphics[width=0.85\textwidth]{3normalization/gtzan_normalization_comparison.png}
        \caption{GTZAN: Before (blue) vs After (coral) normalization - shape preserved, scale standardized}
    \end{figure}
    
    
    \begin{itemize}
        \footnotesize
        \item \textbf{Before}: Wide range of scales (0-5000+)
        \item \textbf{After}: Centered at 0, normalized variance (-3 to +3)
        \item \textbf{Shape}: Distribution patterns preserved
    \end{itemize}
\end{frame}

\begin{frame}{PCA: Dimensionality Reduction}
    \begin{columns}[T]
        \column{0.48\textwidth}
        \textbf{Why PCA?}
        \begin{itemize}
            \footnotesize
            \item 69 dimensions = curse of dimensionality
            \item Correlated features = redundancy
            \item Computational efficiency
            \item Preserve 95\% variance
        \end{itemize}
        
        \vspace{0.3cm}
        
        \begin{table}[h]
            \centering
            \scriptsize
            \begin{tabular}{lcc}
                \toprule
                \textbf{Dataset} & \textbf{69→} & \textbf{Reduction} \\
                \midrule
                GTZAN & 39 & 43.5\% \\
                FMA Small & 45 & 35.7\% \\
                FMA Medium & 45 & 35.7\% \\
                Ludwig & 42 & 39.1\% \\
                Indian & 40 & 42.0\% \\
                \midrule
                \textbf{Avg} & \textbf{42} & \textbf{39.2\%} \\
                \bottomrule
            \end{tabular}
        \end{table}
        
        \column{0.48\textwidth}
        \textbf{Mathematical Foundation}
        
        \begin{equation*}
            \mathbf{C} = \frac{1}{n-1}\mathbf{X}^T\mathbf{X}
        \end{equation*}
        
        \begin{equation*}
            \mathbf{X}_{reduced} = \mathbf{X}\mathbf{V}_k
        \end{equation*}
        
        \vspace{0.2cm}
        
        \begin{block}{Results}
            \footnotesize
            \begin{itemize}
                \item \textbf{Variance retained}: 95.15\% avg
                \item \textbf{Speedup}: 2.7× in K-Means
                \item \textbf{Storage}: 39.2\% reduction
                \item \textbf{Quality}: No information loss
            \end{itemize}
        \end{block}
    \end{columns}
\end{frame}

\begin{frame}{PCA Explained Variance}
    \begin{columns}
        \column{0.48\textwidth}
        \begin{figure}
            \centering
            \includegraphics[width=\textwidth]{4pca/gtzan_explained_variance.png}
            \caption{GTZAN: 39 components reach 95.05\%}
        \end{figure}
        
        \column{0.48\textwidth}
        \begin{figure}
            \centering
            \includegraphics[width=\textwidth]{4pca/indian_music_explained_variance.png}
            \caption{Indian: 40 components reach 95.30\%}
        \end{figure}
    \end{columns}
    
    \vspace{0.2cm}
    
    \begin{itemize}
        \footnotesize
        \item \textbf{PC1}: Captures 16-23\% variance (dominant)
        \item \textbf{Top 10 PCs}: Explain ~60-65\% total variance
        \item \textbf{Rapid accumulation}: Steep initial slope confirms effectiveness
    \end{itemize}
\end{frame}

\begin{frame}{PCA 2D Visualization}
    \begin{columns}
        \column{0.48\textwidth}
        \begin{figure}
            \centering
            \includegraphics[width=\textwidth]{4pca/gtzan_pca_2d.png}
            \caption{GTZAN: Classical/metal separation visible}
        \end{figure}
        
        \column{0.48\textwidth}
        \begin{figure}
            \centering
            \includegraphics[width=\textwidth]{4pca/indian_music_pca_2d.png}
            \caption{Indian: Regional genre clusters}
        \end{figure}
    \end{columns}
    
    \vspace{0.2cm}
    
    \begin{itemize}
        \footnotesize
        \item Partial genre separation in 2D space
        \item Classical and Metal genres show clear boundaries
        \item Rock/Blues/Country overlap (similar acoustics)
    \end{itemize}
\end{frame}

%==============================================================================
% SECTION 8: CLUSTERING EXPERIMENTS
%==============================================================================
\section{Clustering Experiments}

\begin{frame}{Clustering Algorithms Evaluated}
    \begin{columns}[T]
        \column{0.48\textwidth}
        \textbf{Algorithm Configurations (k=10)}
        
        \vspace{0.3cm}
        
        \textbf{1. K-Means}
        \begin{itemize}
            \footnotesize
            \item K-Means++ initialization
            \item 300 max iterations
            \item 10 random restarts
        \end{itemize}
        
        \vspace{0.2cm}
        
        \textbf{2. Agglomerative}
        \begin{itemize}
            \footnotesize
            \item Ward linkage
            \item Euclidean distance
            \item Bottom-up hierarchy
        \end{itemize}
        
        \vspace{0.2cm}
        
        \textbf{3. GMM}
        \begin{itemize}
            \footnotesize
            \item Full covariance matrices
            \item EM algorithm (100 iter)
            \item Soft assignments
        \end{itemize}
        
        \column{0.48\textwidth}
        \textbf{4. Spectral Clustering}
        \begin{itemize}
            \footnotesize
            \item 15-neighbor affinity
            \item RBF kernel
            \item ARPACK eigensolver
        \end{itemize}
        
        \vspace{0.3cm}
        
        \begin{block}{Evaluation Metrics}
            \footnotesize
            \textbf{Internal (no labels):}
            \begin{itemize}
                \scriptsize
                \item Silhouette Score
                \item Davies-Bouldin Index
                \item Calinski-Harabasz Score
            \end{itemize}
            
            \textbf{External (with labels):}
            \begin{itemize}
                \scriptsize
                \item Adjusted Rand Index (ARI)
                \item Normalized Mutual Info (NMI)
                \item Purity
            \end{itemize}
        \end{block}
    \end{columns}
\end{frame}

%==============================================================================
% SECTION 9: RESULTS
%==============================================================================
\section{Results}

\begin{frame}{Cross-Dataset Performance Summary}
    \begin{table}[h]
        \centering
        \small
        \begin{tabular}{lccccl}
            \toprule
            \textbf{Dataset} & \textbf{Tracks} & \textbf{Sil.} & \textbf{ARI} & \textbf{Purity} & \textbf{Best} \\
            \midrule
            GTZAN & 999 & 0.088 & 0.225 & 42.9\% & Spectral \\
            FMA Small & 7,996 & 0.046 & 0.107 & 36.8\% & GMM \\
            FMA Medium & 16,986 & 0.070 & 0.219 & 55.2\% & Spectral \\
            Ludwig & 11,293 & 0.078 & 0.132 & 42.7\% & K-Means \\
            Indian & 500 & 0.142 & 0.196 & 53.0\% & Agglom. \\
            \midrule
            \textbf{Average} & -- & \textbf{0.085} & \textbf{0.176} & \textbf{45.9\%} & -- \\
            \bottomrule
        \end{tabular}
    \end{table}
    
    \vspace{0.3cm}
    
    \begin{columns}
        \column{0.48\textwidth}
        \textbf{Key Findings}
        \begin{itemize}
            \footnotesize
            \item \textbf{45.9\% avg purity}: Nearly half of tracks correctly grouped
            \item \textbf{No dominant algorithm}: Dataset characteristics matter
            \item \textbf{Size effect}: Larger datasets show higher purity
        \end{itemize}
        
        \column{0.48\textwidth}
        \textbf{Algorithm Insights}
        \begin{itemize}
            \footnotesize
            \item \textbf{Spectral}: Best for Western collections
            \item \textbf{Agglomerative}: Excels on Indian music
            \item \textbf{K-Means}: Optimal for streaming data
        \end{itemize}
    \end{columns}
\end{frame}

\begin{frame}{GTZAN Clustering Results}
    \begin{columns}
        \column{0.48\textwidth}
        \begin{figure}
            \centering
            \includegraphics[width=\textwidth]{./results/clustering_images/gtzan_k10.png}
            \caption{GTZAN spectral clustering (k=10)}
        \end{figure}
        
        \column{0.48\textwidth}
        \begin{table}[h]
            \centering
            \scriptsize
            \begin{tabular}{lcc}
                \toprule
                \textbf{Algorithm} & \textbf{ARI} & \textbf{Purity} \\
                \midrule
                Spectral & \textbf{0.225} & \textbf{42.9\%} \\
                K-Means & 0.197 & 40.4\% \\
                GMM & 0.190 & 41.1\% \\
                Agglom. & 0.187 & 39.3\% \\
                \bottomrule
            \end{tabular}
        \end{table}
        
        \vspace{0.2cm}
        
        \textbf{Observations}
        \begin{itemize}
            \footnotesize
            \item Spectral clustering best overall
            \item Classical/Metal well-separated
            \item Blues/Rock/Country overlap
        \end{itemize}
    \end{columns}
\end{frame}

\begin{frame}{FMA Medium Clustering Results}
    \begin{columns}
        \column{0.48\textwidth}
        \begin{figure}
            \centering
            \includegraphics[width=\textwidth]{./results/clustering_images/fma_medium_k10.png}
            \caption{FMA Medium spectral clustering}
        \end{figure}
        
        \column{0.48\textwidth}
        \begin{table}[h]
            \centering
            \scriptsize
            \begin{tabular}{lcc}
                \toprule
                \textbf{Algorithm} & \textbf{ARI} & \textbf{Purity} \\
                \midrule
                Spectral & \textbf{0.219} & \textbf{55.2\%} \\
                K-Means & 0.161 & 53.5\% \\
                GMM & 0.136 & 54.8\% \\
                Agglom. & 0.156 & 52.4\% \\
                \bottomrule
            \end{tabular}
        \end{table}
        
        \vspace{0.2cm}
        
        \textbf{Highlights}
        \begin{itemize}
            \footnotesize
            \item \textbf{Highest purity}: 55.2\%
            \item Largest dataset (16,986 tracks)
            \item Rich genre representations
        \end{itemize}
    \end{columns}
\end{frame}

\begin{frame}{Indian Music Clustering Results}
    \begin{columns}
        \column{0.48\textwidth}
        \begin{figure}
            \centering
            \includegraphics[width=\textwidth]{./results/clustering_images/indian_k10.png}
            \caption{Indian music agglomerative clustering}
        \end{figure}
        
        \column{0.48\textwidth}
        \begin{table}[h]
            \centering
            \scriptsize
            \begin{tabular}{lcc}
                \toprule
                \textbf{Algorithm} & \textbf{ARI} & \textbf{Purity} \\
                \midrule
                Agglom. & \textbf{0.196} & \textbf{53.0\%} \\
                GMM & 0.114 & 46.6\% \\
                K-Means & 0.101 & 47.0\% \\
                Spectral & 0.110 & 48.8\% \\
                \bottomrule
            \end{tabular}
        \end{table}
        
        \vspace{0.2cm}
        
        \textbf{Key Insight}
        \begin{itemize}
            \footnotesize
            \item Hierarchical method wins
            \item Cultural distinctiveness
            \item Cross-cultural validation success
        \end{itemize}
    \end{columns}
\end{frame}

\begin{frame}{Cluster-to-Genre Mapping (k=10)}
    \begin{table}[h]
        \centering
        \scriptsize
        \begin{tabular}{clp{5cm}}
            \toprule
            \textbf{Cluster} & \textbf{Genre} & \textbf{Acoustic Characteristics} \\
            \midrule
            0 & Blues & Slow tempo, guitar-dominant, minor keys \\
            1 & Classical & High spectral complexity, low percussiveness \\
            2 & Country & Acoustic instruments, moderate tempo \\
            3 & Disco/Dance & High tempo, strong beat, repetitive \\
            4 & Hip-Hop & Strong bass, rhythmic vocals, 808 drums \\
            5 & Jazz & Complex harmonics, improvisation patterns \\
            6 & Metal & High energy, distorted guitars, fast tempo \\
            7 & Pop & Balanced spectrum, verse-chorus structure \\
            8 & Reggae & Off-beat rhythm, bass-heavy, laid-back \\
            9 & Rock & Guitar-driven, moderate-high energy \\
            \bottomrule
        \end{tabular}
    \end{table}
    
    \vspace{0.2cm}
    
    \begin{itemize}
        \footnotesize
        \item Mapping via \textbf{majority voting} on cluster composition
        \item Achieves \textbf{45.9\% average purity} across all datasets
        \item Demonstrates meaningful unsupervised genre recovery
    \end{itemize}
\end{frame}

%==============================================================================
% CONCLUSION
%==============================================================================
\section{Conclusion}

\begin{frame}{Key Contributions \& Findings}
    \begin{columns}
        \column{0.48\textwidth}
        \textbf{Technical Achievements}
        \begin{itemize}
            \footnotesize
            \item \textbf{99.94\%} feature extraction success
            \item \textbf{99.99\%} data cleanliness
            \item \textbf{39.2\%} dimensionality reduction
            \item \textbf{95.15\%} variance retained
            \item \textbf{45.9\%} average purity
        \end{itemize}
        
        \vspace{0.3cm}
        
        \textbf{Research Contributions}
        \begin{itemize}
            \footnotesize
            \item First cross-cultural study (Western + Indian)
            \item Unified k=10 cluster-genre mapping
            \item Comprehensive 4-algorithm comparison
            \item Reproducible experimental framework
        \end{itemize}
        
        \column{0.48\textwidth}
        \textbf{Key Insights}
        \begin{itemize}
            \footnotesize
            \item No single algorithm dominates
            \item Dataset characteristics determine optimal choice
            \item Spectral: Large Western datasets
            \item Agglomerative: Cultural distinctiveness
            \item Meaningful genre recovery without labels
        \end{itemize}
        
        \vspace{0.3cm}
        
        \begin{block}{Impact}
            \footnotesize
            \begin{itemize}
                \item Streaming platform organization
                \item Recommendation systems
                \item Cultural music preservation
                \item Automated playlist generation
            \end{itemize}
        \end{block}
    \end{columns}
\end{frame}

\begin{frame}{Future Research Directions}
    \begin{columns}
        \column{0.48\textwidth}
        \textbf{Technical Extensions}
        \begin{itemize}
            \footnotesize
            \item Deep learning embeddings (contrastive learning)
            \item Temporal modeling (RNNs, Transformers)
            \item Semi-supervised refinement
            \item Ensemble clustering methods
        \end{itemize}
        
        \vspace{0.3cm}
        
        \textbf{Dataset Expansion}
        \begin{itemize}
            \footnotesize
            \item Middle Eastern music
            \item African traditional genres
            \item Latin American styles
            \item Cross-dataset transfer learning
        \end{itemize}
        
        \column{0.48\textwidth}
        \textbf{Application Development}
        \begin{itemize}
            \footnotesize
            \item Real-time classification system
            \item Multi-label genre assignment
            \item Web-based interactive explorer
            \item Streaming platform integration
        \end{itemize}
        
        \vspace{0.3cm}
        
        \begin{alertblock}{Research Gaps}
            \footnotesize
            \begin{itemize}
                \item Temporal dynamics preservation
                \item Long-term musical structure
                \item Feature bias toward Western music
                \item Subjective genre boundaries
            \end{itemize}
        \end{alertblock}
    \end{columns}
\end{frame}

%==============================================================================
% REFERENCES
%==============================================================================
\section{References}

\begin{frame}{Key References}
    \scriptsize
    \begin{thebibliography}{99}
        
        \bibitem{singh2024}
        S. Singh et al., ``Identification and clustering of unseen ragas in Indian art music,'' \textit{arXiv:2411.18611}, 2024.
        
        \bibitem{kumar2024}
        R. Kumar et al., ``Enhanced music recommendation systems: K-means clustering approaches,'' \textit{Int. J. Mathematical Engineering}, 2024.
        
        \bibitem{ma2023}
        Y. Ma et al., ``On the effectiveness of speech self-supervised learning for music,'' \textit{ISMIR}, 2023.
        
        \bibitem{wang2023}
        S. Wang et al., ``Self-supervised learning using angular contrastive loss,'' \textit{ICASSP}, 2023.
        
        \bibitem{tzanetakis2002}
        G. Tzanetakis and P. Cook, ``Musical genre classification of audio signals,'' \textit{IEEE Trans. Speech Audio Process.}, 2002.
        
        \bibitem{defferrard2017}
        M. Defferrard et al., ``FMA: A dataset for music analysis,'' \textit{ISMIR}, 2017.
        
        \bibitem{mcfee2015}
        B. McFee et al., ``librosa: Audio and music signal analysis in python,'' \textit{SciPy}, 2015.
        
    \end{thebibliography}
\end{frame}

%==============================================================================
% THANK YOU SLIDE
%==============================================================================
\begin{frame}
    \begin{center}
        {\Huge \textbf{Thank You!}}
        
        \vspace{1cm}
        
        {\Large Questions?}
        
        \vspace{1cm}
        
        {\normalsize
        Anirudh Sharma \\
        22dcs002@nith.ac.in
        
        \vspace{0.5cm}
        
        Machine Learning Assignment (CS-652) \\
        Semester-7 (2025)}
        
        \vspace{1cm}
        
        {\small
        \textbf{GitHub:} \url{https://github.com/anisharma07/music-genere-presentation}}
    \end{center}
\end{frame}

\end{document}
